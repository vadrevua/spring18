\documentclass[journal, a4paper]{IEEEtran}

% modified by Bridget McInnes for CS416 Introduction to NLP

% some very useful LaTeX packages include:

%\usepackage{cite}      % Written by Donald Arseneau
                        % V1.6 and later of IEEEtran pre-defines the format
                        % of the cite.sty package \cite{} output to follow
                        % that of IEEE. Loading the cite package will
                        % result in citation numbers being automatically
                        % sorted and properly "ranged". i.e.,
                        % [1], [9], [2], [7], [5], [6]
                        % (without using cite.sty)
                        % will become:
                        % [1], [2], [5]--[7], [9] (using cite.sty)
                        % cite.sty's \cite will automatically add leading
                        % space, if needed. Use cite.sty's noadjust option
                        % (cite.sty V3.8 and later) if you want to turn this
                        % off. cite.sty is already installed on most LaTeX
                        % systems. The latest version can be obtained at:
                        % http://www.ctan.org/tex-archive/macros/latex/contrib/supported/cite/

\usepackage{graphicx}   % Written by David Carlisle and Sebastian Rahtz
                        % Required if you want graphics, photos, etc.
                        % graphicx.sty is already installed on most LaTeX
                        % systems. The latest version and documentation can
                        % be obtained at:
                        % http://www.ctan.org/tex-archive/macros/latex/required/graphics/
                        % Another good source of documentation is "Using
                        % Imported Graphics in LaTeX2e" by Keith Reckdahl
                        % which can be found as esplatex.ps and epslatex.pdf
                        % at: http://www.ctan.org/tex-archive/info/

%\usepackage{psfrag}    % Written by Craig Barratt, Michael C. Grant,
                        % and David Carlisle
                        % This package allows you to substitute LaTeX
                        % commands for text in imported EPS graphic files.
                        % In this way, LaTeX symbols can be placed into
                        % graphics that have been generated by other
                        % applications. You must use latex->dvips->ps2pdf
                        % workflow (not direct pdf output from pdflatex) if
                        % you wish to use this capability because it works
                        % via some PostScript tricks. Alternatively, the
                        % graphics could be processed as separate files via
                        % psfrag and dvips, then converted to PDF for
                        % inclusion in the main file which uses pdflatex.
                        % Docs are in "The PSfrag System" by Michael C. Grant
                        % and David Carlisle. There is also some information
                        % about using psfrag in "Using Imported Graphics in
                        % LaTeX2e" by Keith Reckdahl which documents the
                        % graphicx package (see above). The psfrag package
                        % and documentation can be obtained at:
                        % http://www.ctan.org/tex-archive/macros/latex/contrib/supported/psfrag/

%\usepackage{subfigure} % Written by Steven Douglas Cochran
                        % This package makes it easy to put subfigures
                        % in your figures. i.e., "figure 1a and 1b"
                        % Docs are in "Using Imported Graphics in LaTeX2e"
                        % by Keith Reckdahl which also documents the graphicx
                        % package (see above). subfigure.sty is already
                        % installed on most LaTeX systems. The latest version
                        % and documentation can be obtained at:
                        % http://www.ctan.org/tex-archive/macros/latex/contrib/supported/subfigure/

\usepackage{url}        % Written by Donald Arseneau
                        % Provides better support for handling and breaking
                        % URLs. url.sty is already installed on most LaTeX
                        % systems. The latest version can be obtained at:
                        % http://www.ctan.org/tex-archive/macros/latex/contrib/other/misc/
                        % Read the url.sty source comments for usage information.

%\usepackage{stfloats}  % Written by Sigitas Tolusis
                        % Gives LaTeX2e the ability to do double column
                        % floats at the bottom of the page as well as the top.
                        % (e.g., "\begin{figure*}[!b]" is not normally
                        % possible in LaTeX2e). This is an invasive package
                        % which rewrites many portions of the LaTeX2e output
                        % routines. It may not work with other packages that
                        % modify the LaTeX2e output routine and/or with other
                        % versions of LaTeX. The latest version and
                        % documentation can be obtained at:
                        % http://www.ctan.org/tex-archive/macros/latex/contrib/supported/sttools/
                        % Documentation is contained in the stfloats.sty
                        % comments as well as in the presfull.pdf file.
                        % Do not use the stfloats baselinefloat ability as
                        % IEEE does not allow \baselineskip to stretch.
                        % Authors submitting work to the IEEE should note
                        % that IEEE rarely uses double column equations and
                        % that authors should try to avoid such use.
                        % Do not be tempted to use the cuted.sty or
                        % midfloat.sty package (by the same author) as IEEE
                        % does not format its papers in such ways.

\usepackage{amsmath}    % From the American Mathematical Society
                        % A popular package that provides many helpful commands
                        % for dealing with mathematics. Note that the AMSmath
                        % package sets \interdisplaylinepenalty to 10000 thus
                        % preventing page breaks from occurring within multiline
                        % equations. Use:
%\interdisplaylinepenalty=2500
                        % after loading amsmath to restore such page breaks
                        % as IEEEtran.cls normally does. amsmath.sty is already
                        % installed on most LaTeX systems. The latest version
                        % and documentation can be obtained at:
                        % http://www.ctan.org/tex-archive/macros/latex/required/amslatex/math/



% Other popular packages for formatting tables and equations include:

%\usepackage{array}
% Frank Mittelbach's and David Carlisle's array.sty which improves the
% LaTeX2e array and tabular environments to provide better appearances and
% additional user controls. array.sty is already installed on most systems.
% The latest version and documentation can be obtained at:
% http://www.ctan.org/tex-archive/macros/latex/required/tools/

% V1.6 of IEEEtran contains the IEEEeqnarray family of commands that can
% be used to generate multiline equations as well as matrices, tables, etc.

% Also of notable interest:
% Scott Pakin's eqparbox package for creating (automatically sized) equal
% width boxes. Available:
% http://www.ctan.org/tex-archive/macros/latex/contrib/supported/eqparbox/

% *** Do not adjust lengths that control margins, column widths, etc. ***
% *** Do not use packages that alter fonts (such as pslatex).         ***
% There should be no need to do such things with IEEEtran.cls V1.6 and later.


% Your document starts here!
\begin{document}

% Define document title and author
	\title{Reading Assignment 2\\Vocabulary Changes in Agatha Christie’s Mysteries as an Indication of Dementia\\ Ian Lancashire and Graeme Hirst 2009}
	\author{Aditya Vadrevu}{}
	\maketitle
% Description of the Study
\section{Description of the Study} 
Ian Lancashire and Graeme Hirst wrote this paper about the late author Agatha Christie and how her dementia affected her writing. The main purpose of this article is to show how dementia affects ones ability to properly communicate with others. According to their research people with Alzheimer's use more indefinite words and repeat words more often than people who are healthy and the same age. The main research question that the authors have is does dementia and other Alzheimer's diseases cause a significant decline in vocabulary size and an increase in repeated phrases and indefinite words.

%Methods and Design of the Study
\section{Methods and Design}
The way that they obtain the data is optimal for what they were trying to achieve. The authors discuss outliers and future work they are going to do to validate their hypothesis even further. The authors also said that all of the novels they analyzed each contain between $55,000$ and $75,000$ words. The way the authors analyzed the books were as follows:
\begin{itemize}
\item They analyzed the first $50,000$ words of each novel
\item Measured vocabulary size which is a count of different words that Christie used
\item Measured vocabulary richness which are defined by the length of the words and how often the word was used
\end{itemize}
The sample size is representative of her condition because it covers a wide range of her age as well as the robustness of the vocabulary used in the books. 
The authors clearly described the entire process they went through to find and process all the data they acquired. I believe that the work that the authors have done can be replicated because they laid out the entire process and how they obtained all their data.

% How the study was Evaluated
\section{Analysis}
They analyzed 16 of Christie's books from when she was 28 to when she was 82. The max amount of wordtypes was a count of 5576 when she was 32, and the lowest was 3762 when she was 81. Christie's actual highest wordtype book was actually 5583 when she was 79 years old but that was not a mystery book she wrote by herself, it was a thriller that was written with some research from a book. The data was appropriate for the hypothesis they formed. The peak of her work in terms of using high wordtypes and low indefinite words was between the ages of 59-63. She had a high word type but relatively high indefinite words where the percentage of words was about $0.4\%$ of total words.

% The results of the Study
\section{Results}
The results from the work of the authors clearly show that through the lifetime of Agatha Christie her lexicon was severely diminished by her dementia. Vocabulary size for her last 3 books were the lowest and the book she wrote when she was 81 was almost $31\%$ lower than a book she wrote 18 years earlier. In this case her book Frankfurt is an outlier because not only was the book written with the help of a research book, but also because she borrowed most of the vocabulary from the book she referenced. The amount of repeated phrases and indefinite words also imply a decline with her age. 

% Limitations of the Study
\section{Limitations}

The main limitations of the analysis done by the authors is that they didn't use the entire book that she wrote. Another limitation that they have is that they only use the books that Christie wrote herself and didn't compare it to a control group. There might be a significant drop in all of three categories with an increase in age for all individuals.

% Conclusion of the Study
\section{Conclusion}

The article was very well written with extremely indepth analysis of how and why the authors worked with the data they were presented. They also listed what they could do to improve on their work as well as outliers which might throw off the values. I feel like they could improve the article by adding a control group to the article before they published it. As of right now it is clear that Christie did have diminishing communication in her books but we aren't sure the hypothesis can be confirmed. They also added one book that contained research and said it was an outlier for all but one section. I did not understand why they would choose to add that book if it contained that much of variance. I also did not agree with the use of the first 50,000 words. I would have done something different like use a random selection of words or just use the whole book and normalize the values. The authors should have done more work before publishing this article.

% Bibliography please include the paper that is being reviewed and any additional
% sources of information
\begin{thebibliography}{5}

	%Each item starts with a \bibitem{reference} command and the details thereafter.
	\bibitem{LH09} Lancashire, Ian, and Hirst, Graeme . “Vocabulary Changes in Agatha Christie’s Mysteries as an Indication of Dementia: A Case Study.” 2009.
\end{thebibliography}

% Your document ends here!
\end{document}