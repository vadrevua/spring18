\documentclass[journal, a4paper]{IEEEtran}

% modified by Bridget McInnes for CS416 Introduction to NLP

% some very useful LaTeX packages include:

%\usepackage{cite}      % Written by Donald Arseneau
                        % V1.6 and later of IEEEtran pre-defines the format
                        % of the cite.sty package \cite{} output to follow
                        % that of IEEE. Loading the cite package will
                        % result in citation numbers being automatically
                        % sorted and properly "ranged". i.e.,
                        % [1], [9], [2], [7], [5], [6]
                        % (without using cite.sty)
                        % will become:
                        % [1], [2], [5]--[7], [9] (using cite.sty)
                        % cite.sty's \cite will automatically add leading
                        % space, if needed. Use cite.sty's noadjust option
                        % (cite.sty V3.8 and later) if you want to turn this
                        % off. cite.sty is already installed on most LaTeX
                        % systems. The latest version can be obtained at:
                        % http://www.ctan.org/tex-archive/macros/latex/contrib/supported/cite/

\usepackage{graphicx}   % Written by David Carlisle and Sebastian Rahtz
                        % Required if you want graphics, photos, etc.
                        % graphicx.sty is already installed on most LaTeX
                        % systems. The latest version and documentation can
                        % be obtained at:
                        % http://www.ctan.org/tex-archive/macros/latex/required/graphics/
                        % Another good source of documentation is "Using
                        % Imported Graphics in LaTeX2e" by Keith Reckdahl
                        % which can be found as esplatex.ps and epslatex.pdf
                        % at: http://www.ctan.org/tex-archive/info/

%\usepackage{psfrag}    % Written by Craig Barratt, Michael C. Grant,
                        % and David Carlisle
                        % This package allows you to substitute LaTeX
                        % commands for text in imported EPS graphic files.
                        % In this way, LaTeX symbols can be placed into
                        % graphics that have been generated by other
                        % applications. You must use latex->dvips->ps2pdf
                        % workflow (not direct pdf output from pdflatex) if
                        % you wish to use this capability because it works
                        % via some PostScript tricks. Alternatively, the
                        % graphics could be processed as separate files via
                        % psfrag and dvips, then converted to PDF for
                        % inclusion in the main file which uses pdflatex.
                        % Docs are in "The PSfrag System" by Michael C. Grant
                        % and David Carlisle. There is also some information
                        % about using psfrag in "Using Imported Graphics in
                        % LaTeX2e" by Keith Reckdahl which documents the
                        % graphicx package (see above). The psfrag package
                        % and documentation can be obtained at:
                        % http://www.ctan.org/tex-archive/macros/latex/contrib/supported/psfrag/

%\usepackage{subfigure} % Written by Steven Douglas Cochran
                        % This package makes it easy to put subfigures
                        % in your figures. i.e., "figure 1a and 1b"
                        % Docs are in "Using Imported Graphics in LaTeX2e"
                        % by Keith Reckdahl which also documents the graphicx
                        % package (see above). subfigure.sty is already
                        % installed on most LaTeX systems. The latest version
                        % and documentation can be obtained at:
                        % http://www.ctan.org/tex-archive/macros/latex/contrib/supported/subfigure/

\usepackage{url}        % Written by Donald Arseneau
                        % Provides better support for handling and breaking
                        % URLs. url.sty is already installed on most LaTeX
                        % systems. The latest version can be obtained at:
                        % http://www.ctan.org/tex-archive/macros/latex/contrib/other/misc/
                        % Read the url.sty source comments for usage information.

%\usepackage{stfloats}  % Written by Sigitas Tolusis
                        % Gives LaTeX2e the ability to do double column
                        % floats at the bottom of the page as well as the top.
                        % (e.g., "\begin{figure*}[!b]" is not normally
                        % possible in LaTeX2e). This is an invasive package
                        % which rewrites many portions of the LaTeX2e output
                        % routines. It may not work with other packages that
                        % modify the LaTeX2e output routine and/or with other
                        % versions of LaTeX. The latest version and
                        % documentation can be obtained at:
                        % http://www.ctan.org/tex-archive/macros/latex/contrib/supported/sttools/
                        % Documentation is contained in the stfloats.sty
                        % comments as well as in the presfull.pdf file.
                        % Do not use the stfloats baselinefloat ability as
                        % IEEE does not allow \baselineskip to stretch.
                        % Authors submitting work to the IEEE should note
                        % that IEEE rarely uses double column equations and
                        % that authors should try to avoid such use.
                        % Do not be tempted to use the cuted.sty or
                        % midfloat.sty package (by the same author) as IEEE
                        % does not format its papers in such ways.

\usepackage{amsmath}    % From the American Mathematical Society
                        % A popular package that provides many helpful commands
                        % for dealing with mathematics. Note that the AMSmath
                        % package sets \interdisplaylinepenalty to 10000 thus
                        % preventing page breaks from occurring within multiline
                        % equations. Use:
%\interdisplaylinepenalty=2500
                        % after loading amsmath to restore such page breaks
                        % as IEEEtran.cls normally does. amsmath.sty is already
                        % installed on most LaTeX systems. The latest version
                        % and documentation can be obtained at:
                        % http://www.ctan.org/tex-archive/macros/latex/required/amslatex/math/



% Other popular packages for formatting tables and equations include:

%\usepackage{array}
% Frank Mittelbach's and David Carlisle's array.sty which improves the
% LaTeX2e array and tabular environments to provide better appearances and
% additional user controls. array.sty is already installed on most systems.
% The latest version and documentation can be obtained at:
% http://www.ctan.org/tex-archive/macros/latex/required/tools/

% V1.6 of IEEEtran contains the IEEEeqnarray family of commands that can
% be used to generate multiline equations as well as matrices, tables, etc.

% Also of notable interest:
% Scott Pakin's eqparbox package for creating (automatically sized) equal
% width boxes. Available:
% http://www.ctan.org/tex-archive/macros/latex/contrib/supported/eqparbox/

% *** Do not adjust lengths that control margins, column widths, etc. ***
% *** Do not use packages that alter fonts (such as pslatex).         ***
% There should be no need to do such things with IEEEtran.cls V1.6 and later.


% Your document starts here!
\begin{document}

% Define document title and author
	\title{Part-of-Speech Tagging for Twitter: Annotation, Features, and Experiments\\Kevin Gimpel, Nathan Schneider, Brendan O'Connor, Dipanjan Das, Daniel Mills, Jacob Eisenstein, Michael Heilman, Dani Yogatama, Jeffrey Flanigan, Noah A. Smith 2011}
	\author{Aditya Vadrevu}{}
	\maketitle
% Description of the Study
\section{Description of the Study} 
In this paper the main goal for the authors is to tag part of speech for users tweets. The authors collected a total of 2217 tweets but cut it down by 390 to only tag tweets that are in English. The authors first created a POS tag system for the tweets and then the authors manually tagged $1827$ tweets. Tagging took about 200 person-hours which was split up by 17 people. This overall project took about two months overall. This problem is more difficult than part of speech tagging a news article or a book because of the various ways a person can convey a message in 140 characters. 

%Methods and Design of the Study
\section{Methods and Design}
This problem that the authors are trying to solve is very difficult not only because of spelling differences but twitter specific grammar and symbols. For example words like LOL or ILY and also shortening words like ima for I'm going to make tagging part of speech more difficult than pure English. The authors also had to take into account the twitter specific characters like the $@$ symbol and Hashtags(\#). First they came up with a tagset which includes tags for punctuation, urls, emojis, as well as normal part of speech tags like nouns, verbs, adjectives, etc. They collected the tweets and ran a WSJ-trained Stanford part of speech tagger to speed up the annotations that they would eventually had to do themselves. Next they pruned the tweets that were not in English and manually tagged the rest of the tweets. After they refined the POS tagger and added hashtags and at-mention functionality. Next They used other features to expand their data. They used TwOrth, Names, TagDict, DistSim, and Metaph.

% How the study was Evaluated
\section{Analysis} 
The authors randomly split the data into 3 parts, training data of ~14,500 tokens, development data with 4,700, and testing data that contains 7,124 tokens. The full feature set achieves an accuracy of 89.37 for the test set. The system predicts verbs, pronouns,prepositions very well and predicts urls and $@$ with almost 100\% accuracy. the system struggles with proper nouns and also the tag G. Even though they were unsuccessful getting the miscellaneous token to get an accuracy above 50\% the authors are happy with the overall system. 


% The results of the Study
\section{Results}
The basic Stanford training was 85.85\% which was better than the basic tagger which they had created without other features. Out of the features Names had the highest test accuracy of 89.39\% and the lowest was DistSim and TagDict. The tags that most accurate were the Url, punctuation, coordinating conjunction and pronouns. All of these achieved an accuracy of 97\% or higher. The lowest accuracy tag, which was 26\%, was the G tag that looks at abbreviations of words.

% Limitations of the Study
\section{Limitations}
The main limitation that they have for this project is that they can not properly identify tags such as ILY and LOL. It is a difficult task but they created a new part of speech tagging system specifically for twitter. The authors could have figured out a better way to split the G tag into smaller more indepth tags. Another hindering limitation is the Metaph system. The Metaph system takes in the phonetic pronunciation of the word and maps it to a simpler word. Words that are almost the same but have different meaning are also roped into one word. For example, War worry and were all sound a bit similar so this system will automatically tag them as the same word.

% Conclusion of the Study
\section{Conclusion}
	I think overall this project was a success but not incredibly ground breaking. Even though their work was done well, I believe that a major part of twitter is people who write solely with acronyms and emojis. Not being able to get a decent accuracy for those tags is disappointing but overall getting hastags, URLs, Emojis and $@$ are a great start and with more learning they can improve upon their original accuracy when predicting a miscellaneous word tag.
% Bibliography please include the paper that is being reviewed and any additional
% sources of information
\begin{thebibliography}{5}

	%Each item starts with a \bibitem{reference} command and the details thereafter.
	\bibitem{GSODMEHYFS11} % Journal paper
	Gimpel, Kevin, et al. “Part of speech tagging for Twitter: annotation, features, experiments.” June 2011.
\end{thebibliography}
% Your document ends here!
\end{document}