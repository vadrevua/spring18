\documentclass[journal, a4paper]{IEEEtran}

% modified by Bridget McInnes for CS416 Introduction to NLP

% some very useful LaTeX packages include:

%\usepackage{cite}      % Written by Donald Arseneau
                        % V1.6 and later of IEEEtran pre-defines the format
                        % of the cite.sty package \cite{} output to follow
                        % that of IEEE. Loading the cite package will
                        % result in citation numbers being automatically
                        % sorted and properly "ranged". i.e.,
                        % [1], [9], [2], [7], [5], [6]
                        % (without using cite.sty)
                        % will become:
                        % [1], [2], [5]--[7], [9] (using cite.sty)
                        % cite.sty's \cite will automatically add leading
                        % space, if needed. Use cite.sty's noadjust option
                        % (cite.sty V3.8 and later) if you want to turn this
                        % off. cite.sty is already installed on most LaTeX
                        % systems. The latest version can be obtained at:
                        % http://www.ctan.org/tex-archive/macros/latex/contrib/supported/cite/

\usepackage{graphicx}   % Written by David Carlisle and Sebastian Rahtz
                        % Required if you want graphics, photos, etc.
                        % graphicx.sty is already installed on most LaTeX
                        % systems. The latest version and documentation can
                        % be obtained at:
                        % http://www.ctan.org/tex-archive/macros/latex/required/graphics/
                        % Another good source of documentation is "Using
                        % Imported Graphics in LaTeX2e" by Keith Reckdahl
                        % which can be found as esplatex.ps and epslatex.pdf
                        % at: http://www.ctan.org/tex-archive/info/

%\usepackage{psfrag}    % Written by Craig Barratt, Michael C. Grant,
                        % and David Carlisle
                        % This package allows you to substitute LaTeX
                        % commands for text in imported EPS graphic files.
                        % In this way, LaTeX symbols can be placed into
                        % graphics that have been generated by other
                        % applications. You must use latex->dvips->ps2pdf
                        % workflow (not direct pdf output from pdflatex) if
                        % you wish to use this capability because it works
                        % via some PostScript tricks. Alternatively, the
                        % graphics could be processed as separate files via
                        % psfrag and dvips, then converted to PDF for
                        % inclusion in the main file which uses pdflatex.
                        % Docs are in "The PSfrag System" by Michael C. Grant
                        % and David Carlisle. There is also some information
                        % about using psfrag in "Using Imported Graphics in
                        % LaTeX2e" by Keith Reckdahl which documents the
                        % graphicx package (see above). The psfrag package
                        % and documentation can be obtained at:
                        % http://www.ctan.org/tex-archive/macros/latex/contrib/supported/psfrag/

%\usepackage{subfigure} % Written by Steven Douglas Cochran
                        % This package makes it easy to put subfigures
                        % in your figures. i.e., "figure 1a and 1b"
                        % Docs are in "Using Imported Graphics in LaTeX2e"
                        % by Keith Reckdahl which also documents the graphicx
                        % package (see above). subfigure.sty is already
                        % installed on most LaTeX systems. The latest version
                        % and documentation can be obtained at:
                        % http://www.ctan.org/tex-archive/macros/latex/contrib/supported/subfigure/

\usepackage{url}        % Written by Donald Arseneau
                        % Provides better support for handling and breaking
                        % URLs. url.sty is already installed on most LaTeX
                        % systems. The latest version can be obtained at:
                        % http://www.ctan.org/tex-archive/macros/latex/contrib/other/misc/
                        % Read the url.sty source comments for usage information.

%\usepackage{stfloats}  % Written by Sigitas Tolusis
                        % Gives LaTeX2e the ability to do double column
                        % floats at the bottom of the page as well as the top.
                        % (e.g., "\begin{figure*}[!b]" is not normally
                        % possible in LaTeX2e). This is an invasive package
                        % which rewrites many portions of the LaTeX2e output
                        % routines. It may not work with other packages that
                        % modify the LaTeX2e output routine and/or with other
                        % versions of LaTeX. The latest version and
                        % documentation can be obtained at:
                        % http://www.ctan.org/tex-archive/macros/latex/contrib/supported/sttools/
                        % Documentation is contained in the stfloats.sty
                        % comments as well as in the presfull.pdf file.
                        % Do not use the stfloats baselinefloat ability as
                        % IEEE does not allow \baselineskip to stretch.
                        % Authors submitting work to the IEEE should note
                        % that IEEE rarely uses double column equations and
                        % that authors should try to avoid such use.
                        % Do not be tempted to use the cuted.sty or
                        % midfloat.sty package (by the same author) as IEEE
                        % does not format its papers in such ways.

\usepackage{amsmath}    % From the American Mathematical Society
                        % A popular package that provides many helpful commands
                        % for dealing with mathematics. Note that the AMSmath
                        % package sets \interdisplaylinepenalty to 10000 thus
                        % preventing page breaks from occurring within multiline
                        % equations. Use:
%\interdisplaylinepenalty=2500
                        % after loading amsmath to restore such page breaks
                        % as IEEEtran.cls normally does. amsmath.sty is already
                        % installed on most LaTeX systems. The latest version
                        % and documentation can be obtained at:
                        % http://www.ctan.org/tex-archive/macros/latex/required/amslatex/math/



% Other popular packages for formatting tables and equations include:

%\usepackage{array}
% Frank Mittelbach's and David Carlisle's array.sty which improves the
% LaTeX2e array and tabular environments to provide better appearances and
% additional user controls. array.sty is already installed on most systems.
% The latest version and documentation can be obtained at:
% http://www.ctan.org/tex-archive/macros/latex/required/tools/

% V1.6 of IEEEtran contains the IEEEeqnarray family of commands that can
% be used to generate multiline equations as well as matrices, tables, etc.

% Also of notable interest:
% Scott Pakin's eqparbox package for creating (automatically sized) equal
% width boxes. Available:
% http://www.ctan.org/tex-archive/macros/latex/contrib/supported/eqparbox/

% *** Do not adjust lengths that control margins, column widths, etc. ***
% *** Do not use packages that alter fonts (such as pslatex).         ***
% There should be no need to do such things with IEEEtran.cls V1.6 and later.


% Your document starts here!
\begin{document}

% Define document title and author
	\title{An Analysis of the AskMSR Question-Answering System\\ Eric Brill, Susan Dumais and Michele Banko  July 2002}
	\author{Aditya Vadrevu}{}
	\maketitle
% Description of the Study
\section{Description of the Study} 
The main purpose of this research is to provide a system in which a question is answered correctly. The AskMSR system predicts an answer using the internet as a giant corpus to find data. This problem is significant because it can help us formulate answers to fact based questions and provides a way to properly retrieve data from a large corpus. The main objectives of this research is to find ways to increase accuracy of the question answering system by finding a proper answer to a question as well as predicting when an answer is incorrect and not answering.


%Methods and Design of the Study
\section{Methods and Design}
The architecture that the authors use to model the system is as follows:
\begin{itemize}
\item Query Reformulation
	\begin{itemize}
	\item[-] Query Reformulation is where the question string is rewritten as an answer by taking all the non stop words and AND-ing them to produce an answer-like statement. For example, "Where is the Eiffel Tower located?" into "The Eiffel Tower is located..." The authors currently do not use a part of speech tagger or a parser to do this but have created rules and weights to do this task.  
	\end{itemize}
\item N-gram Mining
	\begin{itemize}
	\item[-] Using the rewritten questions the system then sends the question to a search engine and gets the results. Using the results that the search engine gives, not the full page responses, the system generates uni-gram, bi-gram, and tri-gram models. The models are then weighted by the question that retrieved that answer as well as the n-gram it uses.
	\end{itemize}
\item N-gram Filtering
	\begin{itemize}
	\item[-] Using one of seven question types the n-gram is reweighted. The n-grams are then analyzed and reweighted again based on how many features are relevant to the question being asked. 15 different filters were added to help reweight the n-grams properly.
	\end{itemize}
\item N-gram Tiling
	\begin{itemize}
	\item[-] The final step in this system is to merge similar answers together to form a longer answer. Answers with similar parts will be joined together and n-grams with different answers will be removed from the list of possible answers.
	\end{itemize}
\end{itemize}
This seems like a very efficient process not only because the system reads only search engine results but because the small answers merge to form a longer answer. This makes it more likely to produce a correct answer.

%\item  Is the rationale for the chosen (model) system described and justified?
%\item Are the size and key characteristics of the sample described?
%\item How representative is the sample?
%\item How were the data collected?
%\item Is the data collection clearly described?
%\item Do the authors discuss the reliability and validity of their methods? Do you believe their method is reproducible?


% How the study was Evaluated
\section{Analysis and Results}
There are 2 different data corpus that the system could use, the web and TREC databases. The main way that the system is used is by using Google as the back end search engine because it provided summaries of a website so that the system could access that instead of reading through the whole page. The accuracy of the default system was $61\%$ and they were able to increase this number by knowing when not to answer the question. Removing or modifying any of the 4 parts of the system caused a significant drop in accuracy with no filtering causing the largest drop of $33\%$. All of the parts of the system are necessary in order to achieve a high accuracy to provide a correct answer. The most important feature seems to be the tiling of n-grams. This makes sense because it is taking small and concise answers and adding on similar answers to form a more in-depth and detailed answer. This will most likely lead to an answer that is correct.

% Limitations of the Study
\section{Limitations}
The biggest limitation that the authors have faced is number retrieval. The input for a numerical retrieval question itself is the issue, the authors report. They state that in-order to get numerical answers we must have numerical queries which they don't do in the system. The authors state that only $12\%$ of failures of the system are actually incorrect and the others can be fixed with minor enhancements to the system.  

% Significance of the Study
\section{Significance}
This paper is significant to the Information Retrieval and Information Extraction communities because it helps explore the ways in which data can be manipulated in-order to form answers to queries that are received. This can also be significant for NLP communities because of the way that the query is rewritten. The rewriting of the question into a simpler answer query is arguably one of the most important parts of the whole system.
% Conclusion of the Study
\section{Conclusion}
Overall this paper gives an in-depth insight into the process that the AskMSR system goes through when it needs to find an answer to a query. The authors go into great detail for each step which is very important and outline where their shortcomings were. They also provide next steps in this project expanding from short answers to full fledged information retrieval. The main point of focus for the paper is that they were able to find a way to predict if the question would produce an answer that was accurate enough, and if it did not then the system would not answer the question. They were able to explain this system well enough to say that this paper is worth reading. 

% Bibliography please include the paper that is being reviewed and any additional
% sources of information
\begin{thebibliography}{5}

	%Each item starts with a \bibitem{reference} command and the details thereafter.
	\bibitem{BRILL02} % Journal paper
	Brill, Eric, et al. “An Analysis of the AskMSR Question-Answering System.” Proceedings of the ACL-02 Conference on Empirical Methods in Natural Language Processing - EMNLP 02, July 2002, doi:10.3115/1118693.1118726.
\end{thebibliography}
% Your document ends here!
\end{document}