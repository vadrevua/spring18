\documentclass[a4paper, 11pt]{article}
\usepackage{comment} % enables the use of multi-line comments (\ifx \fi) 
\usepackage{lipsum} %This package just generates Lorem Ipsum filler text. 
\usepackage{fullpage} % changes the margin

\begin{document}
%Header-Make sure you update this information!!!!
\noindent
\large\textbf{Video Lecture Report 1} \hfill \textbf{Aditya Vadrevu} \\
What will a companionable computational agent be like? \hfill Due Date: 02/02/18 \\
Professor. Yorick Wilks\\

%Question 1
\noindent\rule{16cm}{0.4pt}\\
\noindent {\bf [1 point] Brief Bio of Speaker/s (including current title, affiliation, web site plus a brief description of their background and interests).}\\
https://www.oii.ox.ac.uk/people/yorick-wilks\\
Professor Yorick Wilks currently works at the University of Sheffield in the UK where his concentration of study is Artificial Intelligence. Prof. Wilks also is interested in NLP, the future of the internet and the possibilty of companion like interfaces.

%Question 2
\noindent\rule{16cm}{0.4pt}\\
\noindent {\bf [2 points] In 25 words or less (seriously) what was the most important point (in your opinion) of this presentation? }\\
The most important point in this presentation is that language processing should not only be tasks for a computer but also processing emotions.\\
% Question 3
\noindent\rule{16cm}{0.4pt}\\
\noindent {\bf [2 points] List five (5) facts that you learned from this presentation. These should be actual verifiable facts (rather than opinions). Please express them as complete sentences.} 

\begin{itemize}
\item[1.] Couple of Broad approaches to Machine Dialogue. 
	\begin{itemize}
	\item Chatbot
	\item Logic Base Systems
	\item Extentions of speech engineering systems
	\end{itemize}
\item[2.] People are very willing to fall in love with/ form emotional attachments with robots and mechanical things
\item[3.] Tim Berners-Lee not only created the World Wide Web but also something called the semantic web which helps the internet distinguish what it contains.
\item[4.] Embodied Conversation Agents exist and they build relationships through facial expressions
\item[5.] Fred Jelinek created a system which translates English to French without dictionaries or structure of the language.
\end{itemize}

%Question 4
\noindent\rule{16cm}{0.4pt}\\
\noindent {\bf[5 points] Write a 300-500 word summary of the presentation. Please use complete sentences and take us from the start of the talk through the end. What were the main themes in the talk, how were they integrated together, and how do they apply to what you have previous learned (in this class or another class)?}\\
In this lecture Professor Yorick Wilks mainly talks about a companion but discusses more than that. In the beginning he set out to discuss broad topics in natural language processing and machine dialogue. He talks about some advances in machine dialogue and the importance of them. For example, he talks about the semantic web and Fred Jelinek’s translation system from English to French.
\\ 
\large\textbf{Video Lecture Report 1} \hfill \textbf{Aditya Vadrevu} \\
Professor Wilks then goes on to talk about various companions that are being researched as well as developed. He discussed embodied conversational agent projects (ECA’s) where they construct relationships with facial expressions. Professor Wilks then goes on to talk about the system that he is creating which is taking information from the web and presenting it to you like it knows what you are talking about. He describes in Facebook terms as tagging a picture, by using dialogue to tag pictures. By taking the number of faces and the place the photo was taken they can store this information in a companion to help keep a record of the experiences. Next, Professor Wilks discusses who would use these kinds of companions. He believes that they should be used by older people who need the companion because they are lonely or in need of someone to know all their medical information. He also suggests people who just need an assistant to keep track of their needs should use one. Professor Wilks also goes into the idea of the information collected by this companion and who should get access to it after the person passes away. The government, the company, and family are the highest priority of people who should receive this companion. Professor Wilks states that he believes that these companions will act as a sort of autobiography of the person and that a family should get it so they know truly who that person was.




























\end{document}
