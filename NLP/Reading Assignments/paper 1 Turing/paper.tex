\documentclass[journal, a4paper]{IEEEtran}

% modified by Bridget McInnes for CS416 Introduction to NLP

% some very useful LaTeX packages include:

%\usepackage{cite}      % Written by Donald Arseneau
                        % V1.6 and later of IEEEtran pre-defines the format
                        % of the cite.sty package \cite{} output to follow
                        % that of IEEE. Loading the cite package will
                        % result in citation numbers being automatically
                        % sorted and properly "ranged". i.e.,
                        % [1], [9], [2], [7], [5], [6]
                        % (without using cite.sty)
                        % will become:
                        % [1], [2], [5]--[7], [9] (using cite.sty)
                        % cite.sty's \cite will automatically add leading
                        % space, if needed. Use cite.sty's noadjust option
                        % (cite.sty V3.8 and later) if you want to turn this
                        % off. cite.sty is already installed on most LaTeX
                        % systems. The latest version can be obtained at:
                        % http://www.ctan.org/tex-archive/macros/latex/contrib/supported/cite/

\usepackage{graphicx}   % Written by David Carlisle and Sebastian Rahtz
                        % Required if you want graphics, photos, etc.
                        % graphicx.sty is already installed on most LaTeX
                        % systems. The latest version and documentation can
                        % be obtained at:
                        % http://www.ctan.org/tex-archive/macros/latex/required/graphics/
                        % Another good source of documentation is "Using
                        % Imported Graphics in LaTeX2e" by Keith Reckdahl
                        % which can be found as esplatex.ps and epslatex.pdf
                        % at: http://www.ctan.org/tex-archive/info/

%\usepackage{psfrag}    % Written by Craig Barratt, Michael C. Grant,
                        % and David Carlisle
                        % This package allows you to substitute LaTeX
                        % commands for text in imported EPS graphic files.
                        % In this way, LaTeX symbols can be placed into
                        % graphics that have been generated by other
                        % applications. You must use latex->dvips->ps2pdf
                        % workflow (not direct pdf output from pdflatex) if
                        % you wish to use this capability because it works
                        % via some PostScript tricks. Alternatively, the
                        % graphics could be processed as separate files via
                        % psfrag and dvips, then converted to PDF for
                        % inclusion in the main file which uses pdflatex.
                        % Docs are in "The PSfrag System" by Michael C. Grant
                        % and David Carlisle. There is also some information
                        % about using psfrag in "Using Imported Graphics in
                        % LaTeX2e" by Keith Reckdahl which documents the
                        % graphicx package (see above). The psfrag package
                        % and documentation can be obtained at:
                        % http://www.ctan.org/tex-archive/macros/latex/contrib/supported/psfrag/

%\usepackage{subfigure} % Written by Steven Douglas Cochran
                        % This package makes it easy to put subfigures
                        % in your figures. i.e., "figure 1a and 1b"
                        % Docs are in "Using Imported Graphics in LaTeX2e"
                        % by Keith Reckdahl which also documents the graphicx
                        % package (see above). subfigure.sty is already
                        % installed on most LaTeX systems. The latest version
                        % and documentation can be obtained at:
                        % http://www.ctan.org/tex-archive/macros/latex/contrib/supported/subfigure/

\usepackage{url}        % Written by Donald Arseneau
                        % Provides better support for handling and breaking
                        % URLs. url.sty is already installed on most LaTeX
                        % systems. The latest version can be obtained at:
                        % http://www.ctan.org/tex-archive/macros/latex/contrib/other/misc/
                        % Read the url.sty source comments for usage information.

%\usepackage{stfloats}  % Written by Sigitas Tolusis
                        % Gives LaTeX2e the ability to do double column
                        % floats at the bottom of the page as well as the top.
                        % (e.g., "\begin{figure*}[!b]" is not normally
                        % possible in LaTeX2e). This is an invasive package
                        % which rewrites many portions of the LaTeX2e output
                        % routines. It may not work with other packages that
                        % modify the LaTeX2e output routine and/or with other
                        % versions of LaTeX. The latest version and
                        % documentation can be obtained at:
                        % http://www.ctan.org/tex-archive/macros/latex/contrib/supported/sttools/
                        % Documentation is contained in the stfloats.sty
                        % comments as well as in the presfull.pdf file.
                        % Do not use the stfloats baselinefloat ability as
                        % IEEE does not allow \baselineskip to stretch.
                        % Authors submitting work to the IEEE should note
                        % that IEEE rarely uses double column equations and
                        % that authors should try to avoid such use.
                        % Do not be tempted to use the cuted.sty or
                        % midfloat.sty package (by the same author) as IEEE
                        % does not format its papers in such ways.

\usepackage{amsmath}    % From the American Mathematical Society
                        % A popular package that provides many helpful commands
                        % for dealing with mathematics. Note that the AMSmath
                        % package sets \interdisplaylinepenalty to 10000 thus
                        % preventing page breaks from occurring within multiline
                        % equations. Use:
%\interdisplaylinepenalty=2500
                        % after loading amsmath to restore such page breaks
                        % as IEEEtran.cls normally does. amsmath.sty is already
                        % installed on most LaTeX systems. The latest version
                        % and documentation can be obtained at:
                        % http://www.ctan.org/tex-archive/macros/latex/required/amslatex/math/



% Other popular packages for formatting tables and equations include:

%\usepackage{array}
% Frank Mittelbach's and David Carlisle's array.sty which improves the
% LaTeX2e array and tabular environments to provide better appearances and
% additional user controls. array.sty is already installed on most systems.
% The latest version and documentation can be obtained at:
% http://www.ctan.org/tex-archive/macros/latex/required/tools/

% V1.6 of IEEEtran contains the IEEEeqnarray family of commands that can
% be used to generate multiline equations as well as matrices, tables, etc.

% Also of notable interest:
% Scott Pakin's eqparbox package for creating (automatically sized) equal
% width boxes. Available:
% http://www.ctan.org/tex-archive/macros/latex/contrib/supported/eqparbox/

% *** Do not adjust lengths that control margins, column widths, etc. ***
% *** Do not use packages that alter fonts (such as pslatex).         ***
% There should be no need to do such things with IEEEtran.cls V1.6 and later.


% Your document starts here!
\begin{document}

% Define document title and author
	\title{Reading Assignment 1\\Computing Machinery and Intelligence\\ A. M. Turing, 1950}
	\author{Aditya Vadrevu}{}
	\maketitle
% Description of the Study
\section{Description of the Study} 
In this paper Alan Turing talks about the importance of a digital computer vs a human computer, asking a simple question "Can machines think?" He introduces several different takes on this question and how to best answer them. Turing also discusses and refutes some common objections on this subject, from mathematical ideologies to ESP. This question that Turing mentions is very significant for various reasons. It can help us learn the nature of these 'machines' or digital computers and develop an idea of how to properly get these machines to think on their own. Turing's main points of view are listed as follows:
\begin{itemize}
\item The imitation game should be played at the end of the century and then we will have satisfactory evidence.
\item A thinking machine would need to have $10^9$ binary digits to store and retrieve data
\item To imitate an human mind we must consider three things
	\begin{itemize}	
	\item The state of mind at birth
	\item The education that it has been subjected
	\item The experiences that it has been subjected
	\end{itemize}
\end{itemize}

%Methods and Design of the Study
\section{Contrary Views}
Turing talks about the contrary views in this article where he tries to discuss and refute others opinions and arguments. He talked about 9 different ideologies in which he tries to respond to the argument that was presented.
\begin{itemize}
\item[1.] Theological Objection:
 \\ In this argument the ability to think and an immortal soul is something that god gave only to man and women. Thus since only humans were granted this gift that no animal or machine could think. Turing trying to refute gave an example of how humans procreate to make children so humans creating a thinking machine much like a child is acceptable and does not take away from Gods power of creating souls.
\item[2.] `Heads in the sand' Objection:
	\\ This objection is that a thinking machine would be too dangerous for humans and that the "consequences would be too dreadful."(Turing, 444) Turing does not even bother to refute this objection because he feels that the argument is not substantial to require it.
\item[3.] Mathematical Objection:
	\\ The mathematical objection states that there are always limitations for a discrete-state machine and according to G\"{o}del's Theorem the results cannot be proved nor disproved in the system itself. Turing acknowledges that there are limits but states that there is no proof yet but people who believe this objection will be willing to discuss the Imitation Game experiment.
\item[4.] Argument from Consciousness:
	\\ This argument states that emotions and thought are needed to form a machine that is as good as the human mind. According to this argument "Person A believes that A thinks B does not think and B believes that A does not think." (Turing,446) However usually the argument can be resolved by having a conversation on how everyone thinks. Turing also states that these people would also be open to the Imitation game experiment.
\item[5.] Arguments from Various Disabilities:
	\\ The main argument for this objection is that machines may not be able to do a specific task. This is responded by Turing as he states that these are founded on the principle of scientific induction, which is we draw a general conclusion from what we have seen or experienced. He also expressed that limitations in machines occur because of the small storage capacity. When multiple machines are constructed for a single different purpose they are essentially useless.
\item[6.] Lady Lovelace's Objection:
	\\ Lady Lovelace wrote a memoir about Babbage's Analytical Engine, in which she states that it could only perform things that they knew how to implement. It could not originate or learn anything for itself it needed to be taught everything it could do. Turing then decides to quote Hartree who stated that it does not imply that it impossible for a machine to think for itself but also that machines that were created in that time had the property. Turing fully agrees with Hartree on this point. He implies that there is a possibility that we can construct electronic equipment that can think for itself.
\item[7.] Argument from Continuity in the Nervous System
	\\ This argument states that the nervous system cant be emulated by using a discrete-state system. Because the information that is being passed though a neuron could vary in size, and that outgoing impulse may make a large difference leading to a different action. Thus Turing stated that if the conditions of the imitation game were to be followed the machine would not be able to use the advantages of using a continuous machine. Therefore the discrete state system and the continuous machine would be equal.
\item[8.] Argument from Informality of Behavior
	\\ This argument says that there is no way of forming a rule set to describe exactly what a person should do in a situation. Turing states "If each man had a definite set of rules of conduct by which he regulated his life he would be no better than a machine." (Turing, 452) He goes on to say that there are two main ways a man lives, by laws of behavior and nature. If we find such laws using scientific observations then we would be able to properly construct a discrete-state machine which follows these laws.
\item[9.] Argument from Extrasensory Perception
	\\ The final argument that Turing discusses is that of Extra sensory perception or ESP. Machines can not do these kinds of ESP that which few humans are capable of. For Turing this argument is pretty strong but there are some scientific theories that are workable in practice. He suggests that during the Imitation Game Experiment that the test should be done in a telepathy-proof room.
\end{itemize}

% Limitations of the Study
\section{Limitations}

The main limitation of this article is that it is mostly theoretical. Turing stated that at the end of the century the technology would be there for us to actually test the hypothesis but no physical proof came from this study. The foundation for the thoughts behind this article were very solid having defined a digital computer and comparing it to a human computer, but advancements in technology can not be predicted extremely accurately, there could be some fluctuation. The store, execute and control method could become obsolete if a completely new architecture is created thus changing the whole basis for this article. 

% Significance of the Study
\section{Significance}
The main significance of this article is that it gives us a guideline of how to implement a thinking machine. By refuting common objections a layout or plan for a thinking machine to win the imitation game. The game serves as an experiment to find out how well a thinking machine can figure out which person is A or B given all the options. This is still all for research for now but the benefits of a thinking and learning machine are tremendous. 

% Conclusion of the Study
\section{Conclusion}
	For most of this paper I agree with what Turing had to say. The ideology that a learning machine would have to be nurtured and taught like a child really peaked my interest. If one were to spend time growing and nurturing a machine why not spend time with a human child? What major benefits could a machine with the mind of a human possess. I feel the need to strongly disagree with the `Head in the sand objection'. I believe than a machine raised by humans to think like a human would not cause harm, if they are properly trained. Overall I feel that the paper expressed the task of building a learning machine can not be underestimated. Building a mind that replicates the human brain is an extraordinary task but I believe it is possible and we will find out at the end of the century, when we play the Imitation game.

% Bibliography please include the paper that is being reviewed and any additional
% sources of information
\begin{thebibliography}{5}

	%Each item starts with a \bibitem{reference} command and the details thereafter.
	\bibitem{1} % Journal paper
	Turing, Alan M. “Computing Machinery and Intelligence.” Mind, vol. 59, no. 236, Oct. 1950, pp. 433–460.

\end{thebibliography}

\newpage

% Your document ends here!
\end{document}